\documentclass[11pt,a4paper]{jarticle}
\usepackage{graphicx}
\usepackage{listings}
\usepackage{minted}
\usepackage{mdframed}
\usepackage[top=15truemm, bottom=15truemm, left=15truemm, right=15truemm]{geometry}

\newcommand{\mycode}[2]{
\subsection{#1}
\inputminted[linenos, fontfamily=courier, breaklines, mathescape, fontsize=\small]{cpp}{../_includes/code/#2}
}

\begin{document}
\tableofcontents
\newpage
\section{備忘録}

\mycode{テンプレート}{template.cpp}
\mycode{makev chmax chmin}{makev.cpp}
\subsection{ファイルを作成する}
テンプレートを書いたら、bashを開いて、以下のコマンドを打つ。
\begin{minted}[linenos, fontfamily=courier, breaklines, mathescape, fontsize=\small]{bash}
for i in {A..H}; do cp main.cpp $i.cpp; done
\end{minted}
\newpage
\mycode{構文解析}{parse.cpp}
\newpage
\section{使いそうなライブラリ}
\mycode{繰り返し二乗法}{pow.cpp}
\mycode{約数列挙}{divisor.cpp}
\mycode{素数列挙・素因数分解}{primefactor.cpp}
\newpage
\mycode{ダイクストラ法}{dijkstra.cpp}
\mycode{kruskal法(最小全域木)}{kruskal.cpp}
\mycode{Union-Find}{unionfind.cpp}
\mycode{正方行列}{squarematrix.cpp}
\mycode{Dinic法(最大流)}{dinic.cpp}
\mycode{幾何ライブラリ}{geometry.cpp}
\mycode{木上の非再帰dfs}{treedfs.cpp}
\section{まあまあ使いそうなライブラリ}
\mycode{階乗ライブラリ}{factorial.cpp}
\mycode{Binary Indexed Tree (Fenwick Tree)}{binaryindexedtree.cpp}
\mycode{Segment Tree}{segtree.cpp}
\mycode{遅延伝播Segment Tree}{lazysegtree.cpp}
\mycode{modint(固定MOD)}{modint.cpp}
\mycode{行列}{matrix.cpp}
\section{あまり使わなさそうなライブラリ}
\mycode{中国剰余定理}{CRT.cpp}
\mycode{拡張ユークリッドの互除法}{extgcd.cpp}
\mycode{ベルマンフォード法}{bellman_ford.cpp}
\mycode{二部グラフの最大マッチング}{bipartite_matching.cpp}
\mycode{最大独立集合}{independentset.cpp}
\mycode{最近共通祖先(LCA)}{LCA.cpp}
\mycode{Convex-Hull-Trick}{monotoniccht.cpp}
\mycode{スライド最小値}{slidingwindow.cpp}
\mycode{ポテンシャル付きUnion-Find}{weightedunionfind.cpp}
\mycode{modint(実行時MOD)}{modint_arbitrary.cpp}
\mycode{Xor-Shift}{xorshift.cpp}
\end{document}
