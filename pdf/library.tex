\documentclass[twocolumn, autodetect-engine, 8pt]{jarticle}
\usepackage{graphicx}
\usepackage{listings}
\usepackage{minted}
\usepackage{mdframed}
\usepackage{fancyhdr}
\usepackage[top=15truemm, bottom=5truemm, left=5truemm, right=5truemm]{geometry}

\pagestyle{fancy}
\lhead{Sushi (The University of Electro-Communications)}
\rhead{\thepage}
\newcommand{\mycode}[2]{
\subsection{#1}
\inputminted[fontfamily=courier, breaklines, breakanywhere, mathescape, fontsize=\footnotesize]{cpp}{../_includes/code/#2}}
\begin{document}
\tableofcontents
\newpage
\section{備忘録}

\mycode{テンプレート}{template.cpp}
\mycode{makev chmax chmin}{makev.cpp}
\mycode{unique}{uniq.cpp}
\subsection{ファイルを作成する}
テンプレートを書いたら、bashを開いて、以下のコマンドを打つ。
\begin{minted}[linenos, fontfamily=courier, breaklines, mathescape, fontsize=\small]{bash}
for i in {A..X}; do cp main.cpp $i.cpp; done
for i in {A..X}; do echo -e "add_executable($i.exe $i.cpp)" >> CMakeLists.txt; done
\end{minted}

\begin{minted}[linenos, fontfamily=courier, breaklines, mathescape, fontsize=\small]{cmake}
add_compiler_options(-Wall -Wextra -Wshadow -D_GLIBCXX_DEBUG -ftrapv)
\end{minted}
\newpage
\section{文字列}
\mycode{Aho-Corasick法, Trie}{ahocorasick.cpp}
\mycode{Rolling-Hash}{rolling_hash.cpp}
\mycode{Z-Algorithm}{parse.cpp}
\mycode{構文解析}{z-algorithm.cpp}
\section{未分類}
\mycode{木DFS}{treedfs.cpp}
\mycode{全方位木DP}{rerooting.cpp}
\mycode{HL分解}{hld.cpp}
\mycode{kruskal法}{kruskal.cpp}
\mycode{二部グラフ最大マッチング}{bipartite_matching.cpp}
\mycode{Dijkstra法}{dijkstra.cpp}
\mycode{Bellman-Ford法}{bellman_ford.cpp}
\mycode{最大独立集合}{independentset.cpp}
\mycode{LCA}{LCA.cpp}
\mycode{Garner's algorithm}{garner.cpp}
\mycode{MOD逆元(extgcd)}{modinv.cpp}
\mycode{NTT}{ntt.cpp}
\mycode{約数列挙}{divisor.cpp}
\mycode{中国剰余定理}{CRT.cpp}
\mycode{拡張ユークリッド互除法}{extgcd.cpp}
\mycode{高速素因数分解, 高速素数判定}{primefactor_ll.cpp}
\mycode{素因数分解}{primefactor.cpp}
\mycode{素数列挙}{get_prime.cpp}
\mycode{繰り返し二乗法}{pow.cpp}
\mycode{階乗ライブラリ}{factorial.cpp}
\mycode{任意MODFFT}{fft.cpp}
\mycode{任意MODFFT3}{fft3.cpp}
\mycode{高速kitamasa法}{fastkitamasa.cpp}
\mycode{サイコロ}{dice.cpp}
\mycode{modint(固定MOD)}{modint.cpp}
\mycode{modint(実行時MOD)}{modint_arbitrary.cpp}
\mycode{BinaryIndexedTree}{binaryindexedtree.cpp}
\mycode{CHT}{monotoniccht.cpp}
\mycode{Radix-Heap}{radixheap.cpp}
\mycode{SegTree}{segtree.cpp}
\mycode{LazySegTree}{lazysegtree.cpp}
\mycode{SparseTable}{sparsetable.cpp}
\mycode{スライド最小値}{slidingwindow.cpp}
\mycode{Union-Find}{unionfind.cpp}
\mycode{ポテンシャル付きUnion-Find}{weightedunionfind.cpp}
\mycode{行列}{matrix.cpp}
\mycode{正方行列}{squarematrix.cpp}
\mycode{Xor-Shift}{xorshift.cpp}
\mycode{幾何ライブラリ2次元}{geometry.cpp}
\mycode{幾何ライブラリ3次元}{geometry3d.cpp}
\mycode{Dinic(最大流)}{dinic.cpp}
\mycode{Primal-Dual(最小費用流)}{primaldual.cpp}
\end{document}
